\documentclass{article}
\input{~/.local/share/tex/preamble.tex}

% \usepackage[margin=1in]{geometry}
% \linespread{1.3}
% \pagestyle{headings}

% \input{~/Tex/bibtexref}
% \input{~/Tex/preamble_ref}

\title{Basic Exam, Analysis}

\begin{document}

\maketitle

\section{Metric Spaces}

\subsection{Exposition}\label{sub:exposition_ms}

\begin{theorem}
	A metric space \((X,d)\) is compact if and only if every sequence
	in \(X\) contains a subsequence that converges in \(X\), i.e. \(X\)
	is sequentially compact.
\end{theorem}
\begin{proof}
	First, suppose \(X\) is compact.  For the sake of contradiction,
	assume \((x_n)_{n=1}^{\infty}\subseteq X\) is a sequence without a
	convergent subsequence.  Let \(F \coloneqq \bigcup_{i=n}^{\infty}
	x_n\).  Since \(F\) does not contain a convergent sequence, every
	convergent sequence in \(F\) has a limit point in \(F\), and \(F\)
	is therefore closed.  As \(F\) is a subset of \(X\), it follows
	that \(F\) is compact.  However, because each \(x_n\) is not a
	limit point of \(F\), there exists an open subset \(U_n\) of \(F\)
	containing only \(x_n\) for each \(n\in \mathbb{N}\).  So,
	\(\bigcup_{n=1}^{\infty} U_n\) is an open cover of \(F\) without a
	finite subcover, contradicting the compactness of \(F\).  Hence,
	\(X\) contains a subsequence that converges in \(X\).

	Now, suppose \(X\) is sequentially compact but not compact.  Then
\end{proof}

\subsection{Problems}\label{sub:problems_ms}

\begin{problem}[Spring 2009~\#4; Spring 2005~\#13; Spring 2013~\#3]
Let \((X,d)\) be an arbitrary metric space.
\begin{enumerate}
	\item Give a definition of \textit{compactness} of \(X\) involving open covers.
	\item Define \textit{completeness} of \(X\).
	\item Define \textit{connectedness} of \(X\).
	\item Is the set of rational numbers \(\mathbb{Q}\) (with usual metric) connected?  Justify your answer.
	\item Suppose \(X\) is complete.  Show that \(X\) is compact in the sense of (1) if and only if, for each \(r > 0\), \(X\) can be covered by finitely many balls of radius \(r\), i.e. \(X\) is totally bounded.
\end{enumerate}
\end{problem}

\begin{proof}\leavevmode
	\begin{enumerate}
		\item \(X\) is compact if, for each open cover of \(X\), there is a finite subcover of \(X\).
		\item \(X\) is complete if every Cauchy sequence in \(X\) converges in \(X\).
		\item \(X\) is connected if there exist disjoint, non-empty open sets
		      \(A, B\subseteq X\) such that \(A\cup  B = X\).
		\item \(\Q\) is not connected.  Let \(\alpha\in \R\setminus \Q\) be
		      an arbitrary irrational number.  Define the sets \(A, B\subseteq
		      X\) as below:
		      \begin{align*}
			      A & \coloneqq \{x\in \mathbb{Q} : x < \alpha\} & B & \coloneqq \{x\in \mathbb{Q} : x > \alpha\}
			      .\end{align*}
		      It is clear that \(A\cap B = \emptyset\), \(A, B \neq \emptyset\),
		      and \(A\cup B = X\).  Now, for each \(a\in A\), the set \(\{r\in
		      \mathbb{Q} : a < r < \alpha\}\subseteq A\).  So, \(A\) is open.
		      Similarly, for each \(b\in B\), the set \(\{r\in \mathbb{Q} :
		      \alpha < r < b\}\subseteq B\), whence \(B\) is open.  By
		      definition, it follows that \(\mathbb{Q}\) is not connected.
		\item First, suppose \(X\) is compact.  Let \(\mathfrak{B} = \{B_r(p_1),
		      B_r(p_2), \dots\}\) be a collection of open balls in \(X\) of
		      arbitrary radius \(r>0\) such that \(X\subseteq \bigcup_{i\in
			      \mathbb{N}} B_r(p_i)\).  By the compactness of \(X\), there is a
		      finite index set \(I\subseteq \mathbb{N}\) such that \(X\subseteq
		      \bigcup_{i\in I} B_r(p_i)\).  So, \(X\) is totally bounded.
		      Moreover, as \(X\) is compact, any Cauchy sequence in \(X\) has a
		      convergent subsequence, implying that the mother sequence itself
		      converges.

		      Now, suppose \(X\) is totally bounded and complete.  Let
		      \((x_n)_{n=1}^{\infty}\) be a sequence in \(X\).  By the total
		      boundedness of \(X\), there is a finite collection of open balls
		      \(\left\{B\left(p, \frac{1}{k}\right)\right\}_{p\in F}\) for each
		      \(k\in \mathbb{N}\) such that \(X\subseteq \bigcup_{p\in F}
		      B\left(p, \frac{1}{k}\right)\).  By the pigeonhole principle, at
		      least one ball in \(\left\{B(p,\frac{1}{k})\right\}_{p\in F}\) contains
		      infinitely many terms of \((x_n)\).  So, there is a subsequence
		      \(\left(y_{n}^{(1)}\right)_{n=1}^{\infty}\subseteq (x_n)_{n=1}^{\infty}\) such that
		      \(d\left(y_{n}^{(1)}, y_{m}^{(1)}\right) \le 1\) for each
		      \(m,n\in \mathbb{N}\).  Moreover, we may recursively define
		      subsequences \(\left(y_{n}^{(k+1)}\right)_{n=1}^{\infty}\subseteq
		      \left(y_{n}^{(k)}\right)_{n=1}^{\infty}\) satisfying
		      \(d\left(y_{n}^{(k+1)}, y_{m}^{(k+1)}\right) \le \frac{1}{k+1}\) for each \(m,n,k\in
		      \mathbb{N}\).  Consider the subsequence
		      \(\left(y_n^{(n)}\right)_{n=1}^{\infty} \subseteq (x_n)\).  By the Archimedean
		      Property, for each \(\epsilon>0\), there is some \(N\in \mathbb{N}\)
		      such that
		      \[
			      d\left(y_{n}^{(n)}, y_{m}^{(m)}\right) \le \frac{1}{N} < \epsilon
		      \]
		      for each \(n, m \in \{x\in \mathbb{N} : x\ge N\}\).  Thus,
		      \(\left(y_n^{(n)}\right)_{n=1}^{\infty}\) is a Cauchy subsequence, and,
		      by the completeness of \(X\), converges.
	\end{enumerate}
\end{proof}

\begin{problem}[Fall 2004~\#4]
Suppose that \((M, \rho)\) is a metric space, \(x,y\in M\), and that \((x_n)\) is a
sequence in this metric space such that \(x_n\to x\). Prove that \(\rho(x_n ,
y)\to \rho(x, y)\).
\end{problem}
\begin{proof}
	Let \(\epsilon > 0\) be a real number.  We know that there exists \(N\in
	\mathbb{N}\) such that
	\[
		\rho(x_n,x) < \epsilon \text{ for all } n\ge N
		.\]
	By the triangle inequality, we know that
	\[
		\rho(x_n,y) \le \rho(x_n,x) + \rho(x,y) < \epsilon + \rho(x,y) \iff \rho(x_n,y) - \rho(x,y) < \epsilon
		.\]
	If \(\rho(x_n,y) < \rho(x,y)\), then
	\[
		|\rho(x_n, y) - \rho(x,y)| = \rho(x_n,y) - \rho(x,y) < \epsilon
		.\]
	Otherwise, note that, by the triangle inequality,
	\[
		\rho(x,y) \le \rho(x,x_n) + \rho(x_n,y) < \epsilon + \rho(x_n,y) \iff
		\rho(x,y) - \rho(x_n,y) < \epsilon
		.\]
	Hence,
	\[
		|\rho(x_n,y) - \rho(x,y)| < \epsilon
	\]
	for each \(n\ge N\).  Since \(\epsilon > 0\) is arbitrary, \(\rho(x_n,y)\to
	\rho(x,y)\) as \(n\to \infty\).
\end{proof}

\begin{problem}[Fall 2002~\#1]
Let \(K\) be a compact subset and \(F\) be a closed subset in the metric space
\(X\).  Suppose \(K\cap  F = \emptyset\).  Prove that
\[
	0 < \inf_{x\in K, y\in F}d(x,y)
\]
\end{problem}
\begin{proof}
	Assume, to the contrary, that \(\inf \left(\{d(x,y) : x\in K, y\in
	F\}\right) = 0\) (since \(d(x,y) > 0\), the infimum may not be less than 0).
	There exist sequences \((x_n)_{n=1}^{\infty}\subseteq K,
	(y_n)_{n=1}^{\infty}\subseteq F\) such that \(d(x_n,y_n)\leq \frac{1}{n}\)
	for each \(n\in \mathbb{N}\).  Because \(K\) is compact, \((x_n)\) contains a
	subsequence \((x_{n_k})_{k=1}^{\infty}\) such that \(x_{n_k}\to x\) as \(k\to
	\infty\) for some \(x\in K\).  Moreover, the subsequence
	\((y_{n_k})_{n=1}^{\infty}\subseteq (y_n)\) is such that, for each \(\epsilon > 0\), there exists some \(N\in \mathbb{N}\) such that
	\[
		d(y_{n_k}, x) \leq d(y_{n_k}, x_{n_k}) + d(x_{n_k}, x) < \frac{1}{n_k} + \frac{\left(n_k-\frac{1}{\epsilon}\right) \epsilon}{n_k} = \epsilon \text{ if } n_k\ge N
		.\]
	That is, \(y_{n_k}\to x\) as \(k\to \infty\).  Since \(F\) is closed, it
	follows that \(x\in F\).  So, \(x\in K\cap F\), contradicting the fact that
	\(K\cap F = \emptyset\).  Thus, \(\inf \left(\{d(x,y) : x\in K, y\in
	F\}\right) > 0\).
\end{proof}

\begin{problem}[Fall 2008~\#4, Fall 2010~\#1]\leavevmode
\begin{enumerate}
	\item Suppose that \(K\) and \(F\) are subsets of \(\mathbb{R}^2\) with
	      \(K\) closed and bounded and \(F\) closed.  Prove that if \(K\cap F =
	      \emptyset\), then \(d(K,F) > 0\).  Recall that
	      \[
		      d(K,F) = \inf \left( \{d(x,y) : x\in K, y\in F\} \right)
		      .\]
	\item Show the converse, that if \(K\subseteq X\) is compact and
	      \[
		      \inf_{x\in K, y\in F} d(x,y) > 0
		      ,\]
	      then \(K\cap F = \emptyset\).
	\item If \(K\) is merely closed, does (1) hold?
\end{enumerate}
\end{problem}
\begin{proof}\leavevmode
	\begin{enumerate}
		\item Assume, to the contrary, that \(\inf \left( \{d(x,y) : x\in K, y\in
		      F\}\right) = 0\).  Then, there are sequences
		      \((x_n)_{n=1}^{\infty}\subseteq K\) and \((y_n)_{n=1}^{\infty}\subseteq
		      F\) such that \(d(x_n, y_n)\leq \frac{1}{n}\) for each \(n\in
		      \mathbb{N}\).  Since \(K\) is closed and bounded, there is a subsequence
		      \((x_{n_k})_{k=1}^{\infty}\subseteq (x_n)_{n=1}^{\infty}\) such that
		      \(x_{n_k}\to x\) as \(k\to \infty\) for some \(x\in K\) by the
		      Bolzano-Weierstrass Theorem.  The subsequence
		      \((y_{n_k})_{k=1}^{\infty}\subseteq (y_n)_{n=1}^{\infty}\) is
		      guaranteed to converge to \(x\); for each \(\epsilon > 0\), there is
		      some \(N\in \mathbb{N}\) such that
		      \[
			      d(y_{n_k}, x)\leq d(y_{n_k}, x_{n_k}) + d(x_{n_k}, x) < \frac{1}{n_k} + \frac{\left( n_k-\frac{1}{\epsilon}\right) \epsilon}{n_k} = \epsilon \text{ if } n_k\geq N
			      .\]
		      As \(F\) is closed, \(x\in F\), whence \(x\in K\cap F\).  This
		      contradicts the assumption that \(K\cap F = \emptyset\).  Thus,
		      \(d(K,F) > 0\).
		\item Suppose \(K\cap F \neq \emptyset\).  Let \(p\in K\cap F\).  Then
		      \[
			      \inf(\{d(x,y) : x\in K, y\in F\})\le d(p,p) = 0
			      ,\]
		      contradicting \(\inf(\{d(x,y) : x\in K, y\in F\})> 0\).
		\item No, (1) does not necessarily hold.  Let \(K\coloneqq \{(n,0) :
		      n\in \mathbb{N}\}\), and \(F\coloneqq \left\{ \left( n+\frac{1}{2n},
		      0\right) : n\in \mathbb{N} \right\}\) (note that \(0\notin
		      \mathbb{N}\).  \(K, F\) are closed since they contain all their limit
		      points (of which there are none) and disjoint by definition.
	\end{enumerate}
\end{proof}

\begin{problem}[Spring 2002~\#3]
Suppose \(X\) is a (countably) compact metric space.  Prove that every
sequence \((x_n)_{n=1}^{\infty}\subseteq X\) has a convergent subsequence.
\end{problem}
\begin{proof}
	% Let \((V_{\alpha})_{\alpha\in I}\) be a open cover of \(X\).  By the
	% compactness of \(X\), there is a finite subset \(J\subseteq I\) such that
	% \((V_{\alpha})_{\alpha\in J}\) is a (finite) open (sub)-cover of \(X\).  
	Suppose, for the sake of contradiction, \((x_n)_{n=1}^{\infty}\subseteq X\)
	does not have a convergent subsequence.  Then, for each \(x\in X\), there is
	some \(\epsilon > 0\) such that \(B(x,\epsilon)\) contains only finitely many
	\(x_n\).  By the compactness of \(X\), we know that there is a finite cover
	\(\left\{ B(x, \epsilon)\right\}_{x\in I}\) of \(X\).  However, as there are
	only finitely many terms of \(x_n\) in each of finite collection of balls
	\(\{B(x, \epsilon)\}_{x\in I}\), it follows that there are only finitely many
	terms in the sequence \((x_n)\), a contradiction.
\end{proof}

\begin{problem}[Spring 2005~\#12]
Let \((X,d)\) be a metric space.  Prove that the following are equivalent:
\begin{enumerate}
	\item There is a countable dense set.
	\item There is a countable basis for the topology.
\end{enumerate}
Recall that a collection of open sets \(U\) is called a basis if every open set
can be expressed as a union of elements of \(U\).
\end{problem}
\begin{proof}
	First, suppose there is a countable dense set, say \(\{x_n\}_{n=1}^{\infty}\).
	Let
	\[
		\mathcal{B}\coloneqq \{B(x_n,r) : n\in \mathbb{N} \text{ and } r\in
		\mathbb{Q}\}
		.\]
	I claim that \(\mathcal{B}\), which is countable, is a basis.  Let \(U\) be
	an open neighborhood of some \(x\in X\).  We want to show that there exists
	an open set \(B\in \mathcal{B}\) such that \(x\in B\) and \(B\subseteq U\).
	As \(U\) is open, there exists some \(r\in \mathbb{R}\) such that \(B(x,
	r)\subseteq U\); as the \(\mathbb{Q}\) is dense in \(\mathbb{R}\), we may
	assume \(r\in \mathbb{Q}\).  Since \(\{x_n\}_{n=1}^{\infty}\) is dense, there
	exists some \(k\in \mathbb{N}\) such that \(x_{k}\in B \left( x,
	\frac{r}{2}\right)\cap \{x_n\}_{n=1}^{\infty}\).  Letting \(B = B \left( x_k,
	\frac{r}{2} \right)\subseteq B(x, r)\), we see that \(x\in B\) and \(B\subseteq
	U\).

	Conversely, suppose there is a countable basis \(\mathcal{B} =
	\{B_n\}_{n=1}^{\infty}\).  For each \(n\in \mathbb{N}\), select a point
	\(x_n\in B_n\).  The set \(\{x_n\}_{n=1}^{\infty}\) is dense.  Indeed, any
	non-empty open set \(U\subseteq X\) can be expressed as \(U = \bigcup_{i\in
		I} B_{i}\).  Thus, \(\bigcup_{i\in I} x_i\subseteq U\).
\end{proof}

\begin{problem}[Spring 2006~\#6]
Let \(X\) be the set of all infinite sequences
\(\{\sigma_n\}_{n=1}^{\infty}\) of 1's and 0's endowed with the metric
\[
	d \left( \{\sigma_n\}_{n=1}^{\infty}, \{\sigma'_n\}_{n=1}^{\infty} \right) = \sum_{n=1}^{\infty} \frac{1}{2^n} \left|\sigma_n - \sigma'_n \right|
	.\]
Give a direct proof that every infinite subset of \(X\) has an accumulation point.
\end{problem}
\begin{proof}
	Let \(S\) be an infinite subset of \(X\).  By the pigeonhole principle, for
	each \(l\in \mathbb{N}\), there is some sequence
	\(\{\omega_n\}_{n=1}^{\infty}\in X\), such that
	\[
		\omega_n = \sigma_n \text{ for } n\in \{1, \ldots, l\}
	\]
	for infinitely many \(\{\sigma_n\}_{n=1}^{\infty}\in S\); we may choose
	\(\{\omega_n\}_{n=1}^{\infty}\in S\).  Now, let \(r > 0\).  By the Archmedean
	property of the real numbers, there is some \(n\in \mathbb{N}\) such that \(r
	> \frac{1}{2^{n}}\).  By letting \(l = n+1\), we see that
	\[
		\omega_i = \sigma_i \text{ for } i\in \{1, \ldots, n+1\}
	\]
	for infinitely many \(\{\sigma_n\}_{n=1}^{\infty}\in S\), whence
	\[
		d(\omega_i, \sigma_i)\leq \sum_{i=n+1}^{\infty} \frac{1}{2^i} =
		\frac{\frac{1}{2^{n+1}}}{1-\frac{1}{2}} = \frac{1}{2^{n}} < r
		.\]
	Therefore, \(\{\omega_n\}_{n=1}^{\infty}\in S\) is an accumulation point of
	\(S\).
\end{proof}

\begin{problem}[Spring 2005~\#7]
Let \(X,Y\) be two topological spaces.  We say that a continuous function
\(f: X\to Y\) is \textit{proper} if \(f^{-1}(K)\) is compact for any compact
set \(K\subseteq Y\).
\begin{enumerate}
	\item Give an example of a function that is proper but not a homeomorphism.
	\item Give an example of a function that is continuous but not proper.
	\item Suppose \(f:\mathbb{R}\to \mathbb{R}\) is in \(C^1\) (that is, has a
	      continuous derivative) and for all \(x\in \mathbb{R}\),
	      \[
		      \left\lvert f'(x)\right\lvert \geq  1
		      .\]
	      Show that \(f\) is proper.
\end{enumerate}
\end{problem}
\begin{proof}\leavevmode
	\begin{enumerate}
		\item Let \(f: \mathbb{R}\to \mathbb{R}\) be defined by \(f(x)\coloneqq
		      x^2\).  \(f\) is continuous and proper, but it is not a bijection; so, it is
		      not a homeomorphism.
		\item Let \(f: \mathbb{R}\to \mathbb{R}\) be defined by \(f(x)\coloneqq
		      0\).  \(f\) is continuous, but it is not proper.  Indeed, \(\{0\}\)
		      is compact, but \(f^{-1}(0) = \mathbb{R}\) is not.
		\item Let \(K\subseteq \mathbb{R}\) be a compact set.  As \(|f'(x)|\ge 1\)
		      and \(f\in C^1\), it follows that \(f\) must either be strictly
		      increasing or decreasing; otherwise, by the intermediate value
		      theorem, there is a root \(x_0\in \mathbb{R}\) of \(f'\).  So, \(f\)
		      is a continuous bijection.  Now, by the inverse function theorem, we
		      know that
		      \[
			      \left(f^{-1}\right)'(y) = \frac{1}{f' \left(f^{-1}(y)\right)}
			      .\]
		      Since \(|f'(x)|\ge 1\) for each \(x\in \mathbb{R}\),
		      \(\left(f^{-1}\right)'(y)\neq 0\) for each \(y\in \mathbb{R}\).  So,
		      \(\left(f^{-1}\right)'\) is defined over \(\mathbb{R}\), \(f^{-1}\)
		      is continuous, and \(f\) is a homeomorphism.  Finally, since the
		      image of a compact set with respect to a continuous function is
		      compact, \(f\) must be proper.
	\end{enumerate}
\end{proof}

\section{Fixed Point}

\begin{problem}[Spring 2008~\#1]
Let \(g\in C([a,b])\), with \(a\leq g(x)\leq b\) for each \(x\in [a,b]\).
Prove the following:
\begin{enumerate}
	\item \(g\) has at least one fixed point \(p\) in the interval \([a,b]\).
	\item If there is some \(\gamma\in \mathbb{R}\) such that \(\gamma < 1\) and
	      \[
		      |g(x) - g(y)| \leq \gamma |x-y|
	      \]
	      for each \(x,y\in [a,b]\), then the fixed point is unique, and the iteration
	      \[
		      x_{n+1} = g(x_n)
	      \]
	      converges to \(p\) for any initial guess \(x_0\in [a,b]\).
\end{enumerate}
\end{problem}

\begin{proof}\leavevmode
	\begin{enumerate}
		\item Consider the function \(h(x)\coloneqq g(x) - x\).  The point \(x\in
		      [a,b]\) is a fixed point if and only if \(h(x) = 0\).  So, if we assume
		      that \(g\) has no fixed points in \([a,b]\), then \(h(x) > 0\) or
		      \(h(x) < 0\) for each \(x\in [a,b]\) by the continuity of \(h\).  If
		      we assume the former, we have that \(h(b) = g(b) - b > 0 \iff g(b) >
		      b\), contradicting the assumption that \(a\leq g(x)\leq b\) for each
		      \(x\in [a,b]\).  A similar contradiction occurs if we assume \(h(x) <
		      0\) for each \(x\in [a,b]\).
		\item Suppose \(p,q\in [a,b]\), \(p\neq q\) are fixed points of \(g\).  Then
		      \[
			      |g(p) - g(q)| = |p - q| \leq \gamma |p - q|
			      ,\]
		      where \(\gamma < 1\).  Since \(|p-q| > 0\), we may divide both sides
		      by \(|p-q|\) to obtain \(1 \leq \gamma\), a contradiction.  Hence,
		      \(|p-q| = 0\), and the fixed point \(p\in [a,b]\) is unique.

		      Let \(x_0\in [a,b]\) be some initial guess for the fixed point \(p\)
		      of the iteration \(x_{n+1} = g(x_n)\).  I claim that, for each \(n\in \mathbb{N}\),
		      \[
			      |g(x_n) - p| \leq \gamma^{n+1} |x-y|
			      .\]
		      To show this, we proceed by induction.  The case when \(n=0\) is
		      obvious (note that \(g(p) = p\)).  Suppose, now, that \(|g(x_k) - p|
		      \leq \gamma^{k+1} |x-y|\) for some \(k\in \mathbb{N}\).  Then,
		      \[
			      |g(x_{k+1}) - p|
		      \]
	\end{enumerate}
\end{proof}


\end{document}
